\chapter{Discussion}

\paragraph{En conclusion,} ce projet nous a permis de découvrir et de mettre en pratique les techniques d'IA sur microcontrôleurs. 
Au travers de ce projet, nous avons appris à construire un jeu de données, 
et à optimiser un réseau de neurones pour pouvoir l'utiliser dans un système à ressources restreintes.

Nos résultats de détection des différentes espèces d'oiseaux ont été satisfaisants. Cependant, 
l'amélioration de la qualité notre jeu de données aurait pu permettre d'obtenir de meilleurs résultats.

Une fois le projet terminé, nous avons pu continuer à nettoyer notre jeu de données 
en utilisant un algorithme de forêts aléatoires pour marquer les données potentiellement incorrectement labellisées.
Les résultats ont été prometteurs et ont permis d'améliorer légèrement nos résultats tout en ayant
un nombre inférieur d'échantillon. Cette approche couplée à une augmentation des données 
pourraient donner de bons résultats.

Il pourrait être intéressant d'utiliser ces techniques pour améliorer nos jeux de données et ainsi obtenir
de meilleurs résultats lors des tests sur microcontrôleurs.
