\chapter{Introduction}

Le domaine du CERN est un terrain riche en matière de biodiversité et présente plusieurs espèces rares \cite{CernBiodiversity1}.
Dans le cadre d'une politique de la sensibilisation à l'environnement l'Organisation a mis en places plusieurs mesures
visant à favoriser la biodiversité sur ses terrains. 
Une des mesures mise en \oe{}uvre implique des relevés des diverses espèces 
qui composent la faune et la flore du domaine du CERN \cite{CernBiodiversity2}

Dans ce projet, notre objectif sera de proposer un moyen de faciliter le relevé des différentes espèces d'oiseaux
présent sur le domaine du CERN.
Pour cela, notre nous utiliserons un microcontrolleur STM32 NUCLEO-L476RG ainsi qu'un microphone analogique fourni 
par l'Université Côte d'Azur. 
Notre objectif sera de pouvoir différencier 10 espèces d'oiseau d'intérêt potentiellement présentes sur le domaine du CERN.
Pour tester nos résultats, nous vérifierons que les espèces soient correctement reconnues par la carte. Nous effectuerons ces tests
en lançant sur hauts parleurs différents sons d'oiseaux. 
Les sons utilisés pour réaliser ces tests en temps réel seront différents des sons présents dans le jeu de données d'entrainement et de test.
